\documentclass[12pt,letterpaper]{article}
\usepackage{graphicx,textcomp}
\usepackage{natbib}
\usepackage{setspace}
\usepackage{fullpage}
\usepackage{color}
\usepackage[reqno]{amsmath}
\usepackage{amsthm}
\usepackage{fancyvrb}
\usepackage{amssymb,enumerate}
\usepackage[all]{xy}
\usepackage{endnotes}
\usepackage{lscape}
\newtheorem{com}{Comment}
\usepackage{float}
\usepackage{hyperref}
\newtheorem{lem} {Lemma}
\newtheorem{prop}{Proposition}
\newtheorem{thm}{Theorem}
\newtheorem{defn}{Definition}
\newtheorem{cor}{Corollary}
\newtheorem{obs}{Observation}
\usepackage[compact]{titlesec}
\usepackage{dcolumn}
\usepackage{tikz}
\usetikzlibrary{arrows}
\usepackage{multirow}
\usepackage{xcolor}
\newcolumntype{.}{D{.}{.}{-1}}
\newcolumntype{d}[1]{D{.}{.}{#1}}
\definecolor{light-gray}{gray}{0.65}
\usepackage{url}
\usepackage{listings}
\usepackage{color}

\definecolor{codegreen}{rgb}{0,0.6,0}
\definecolor{codegray}{rgb}{0.5,0.5,0.5}
\definecolor{codepurple}{rgb}{0.58,0,0.82}
\definecolor{backcolour}{rgb}{0.95,0.95,0.92}

\lstdefinestyle{mystyle}{
	backgroundcolor=\color{backcolour},   
	commentstyle=\color{codegreen},
	keywordstyle=\color{magenta},
	numberstyle=\tiny\color{codegray},
	stringstyle=\color{codepurple},
	basicstyle=\footnotesize,
	breakatwhitespace=false,         
	breaklines=true,                 
	captionpos=b,                    
	keepspaces=true,                 
	numbers=left,                    
	numbersep=5pt,                  
	showspaces=false,                
	showstringspaces=false,
	showtabs=false,                  
	tabsize=2
}
\lstset{style=mystyle}
\newcommand{\Sref}[1]{Section~\ref{#1}}
\newtheorem{hyp}{Hypothesis}


\title{Problem Set 4}
\date{Due: December 3, 2023}
\author{Applied Stats/Quant Methods 1}


\begin{document}
	\maketitle
	\section*{Instructions}
	\begin{itemize}
		\item Please show your work! You may lose points by simply writing in the answer. If the problem requires you to execute commands in \texttt{R}, please include the code you used to get your answers. Please also include the \texttt{.R} file that contains your code. If you are not sure if work needs to be shown for a particular problem, please ask.
		\item Your homework should be submitted electronically on GitHub.
		\item This problem set is due before 23:59 on Sunday December 3, 2023. No late assignments will be accepted.
	\end{itemize}



	\vspace{.5cm}
\section*{Question 1: Economics}
\vspace{.25cm}
\noindent 	
In this question, use the \texttt{prestige} dataset in the \texttt{car} library. First, run the following commands:

\begin{verbatim}
install.packages(car)
library(car)
data(Prestige)
help(Prestige)
\end{verbatim} 


\noindent We would like to study whether individuals with higher levels of income have more prestigious jobs. Moreover, we would like to study whether professionals have more prestigious jobs than blue and white collar workers.

\newpage
\begin{enumerate}
	
	\item [(a)]
	Create a new variable \texttt{professional} by recoding the variable \texttt{type} so that professionals are coded as $1$, and blue and white collar workers are coded as $0$ (Hint: \texttt{ifelse}).
	
	\vspace{0cm}
	
	\begin{lstlisting}
	# Create a new variable 'professional' by recoding the 'type' variable
	Prestige$professional <- ifelse(Prestige$type %in% c("prof", "bc"), 1, 0)
		
	# View the updated dataset
	head(Prestige)
	\end{lstlisting}
	
	\begin{verbatim}
                    education income women prestige census type professional
gov.administrators      13.11  12351 11.16     68.8   1113 prof            1
general.managers        12.26  25879  4.02     69.1   1130 prof            1
accountants             12.77   9271 15.70     63.4   1171 prof            1
purchasing.officers     11.42   8865  9.11     56.8   1175 prof            1
chemists                14.62   8403 11.68     73.5   2111 prof            1
physicists              15.64  11030  5.13     77.6   2113 prof            1
		
	\end{verbatim}
	
	
	\item [(b)]
	Run a linear model with \texttt{prestige} as an outcome and \texttt{income}, \texttt{professional}, and the interaction of the two as predictors (Note: this is a continuous $\times$ dummy interaction.)
	
	\begin{lstlisting}
	# Run a linear model with prestige as the outcome
	Regression_1 <- lm(prestige ~ income * professional, data = Prestige)

	# Display the summary of the model
	summary(Regression_1)
	\end{lstlisting}
	
	\begin{verbatim}
Coefficients:
Estimate Std. Error t value Pr(>|t|)    
(Intercept)          2.759e+01  5.649e+00   4.884 4.04e-06 ***
income               2.821e-03  1.070e-03   2.636  0.00976 ** 
professional        -5.594e-01  6.273e+00  -0.089  0.92913    
income:professional  8.629e-05  1.114e-03   0.077  0.93844    
---
Signif. codes:  0 ‘***’ 0.001 ‘**’ 0.01 ‘*’ 0.05 ‘.’ 0.1 ‘ ’ 1

Residual standard error: 12.21 on 98 degrees of freedom
Multiple R-squared:  0.5111,	Adjusted R-squared:  0.4962 
F-statistic: 34.15 on 3 and 98 DF,  p-value: 3.386e-15
		
	\end{verbatim}
	
	\vspace{0cm}
	\newpage
	\item [(c)]
	Write the prediction equation based on the result.
\begin{align*}
	Y &= B_0 + B_1X_1 + B_2X_2 + B_3X_3 + E \quad \text{(Multiple Linear Regression Model)} \\
	\text{Prestige} &= B_0 + B_1 \cdot \text{Income} + B_2 \cdot \text{Professional} + B_3 \cdot \text{Income} \cdot \text{Professional} \\
	\text{Prestige} &= 2.759 + 2.821 \cdot \text{Income} - 5.594 \cdot \text{professional} + 8.629 \cdot \text{income:professional}
\end{align*}

	\vspace{1cm}

	\item [(d)]
	Interpret the coefficient for \texttt{income}.

	
	 For each one-unit increase in income, the Prestige is expected to increase by 2.821 units, assuming all other variables are held constant.
	
	\vspace{1cm}	
	
	\item [(e)]
	Interpret the coefficient for \texttt{professional}.
	
	The coefficient of -5.594 indicates that, on average, professionals have a Prestige score that is 5.594 units lower than that of non-professionals, assuming all other variables are held constant.
	
	\vspace{1cm}	
	
	\item [(f)]
	What is the effect of a \$1,000 increase in income on prestige score for professional occupations? In other words, we are interested in the marginal effect of income when the variable \texttt{professional} takes the value of $1$. Calculate the change in $\hat{y}$ associated with a \$1,000 increase in income based on your answer for (c).
	
	\vspace{1cm}
	
	\newpage
	
	\item [(g)]
	What is the effect of changing one's occupations from non-professional to professional when her income is \$6,000? We are interested in the marginal effect of professional jobs when the variable \texttt{income} takes the value of $6,000$. Calculate the change in $\hat{y}$ based on your answer for (c).
	
	
	
	
\end{enumerate}

\newpage

\section*{Question 2: Political Science}
\vspace{.25cm}
\noindent 	Researchers are interested in learning the effect of all of those yard signs on voting preferences.\footnote{Donald P. Green, Jonathan	S. Krasno, Alexander Coppock, Benjamin D. Farrer,	Brandon Lenoir, Joshua N. Zingher. 2016. ``The effects of lawn signs on vote outcomes: Results from four randomized field experiments.'' Electoral Studies 41: 143-150. } Working with a campaign in Fairfax County, Virginia, 131 precincts were randomly divided into a treatment and control group. In 30 precincts, signs were posted around the precinct that read, ``For Sale: Terry McAuliffe. Don't Sellout Virgina on November 5.'' \\

Below is the result of a regression with two variables and a constant.  The dependent variable is the proportion of the vote that went to McAuliff's opponent Ken Cuccinelli. The first variable indicates whether a precinct was randomly assigned to have the sign against McAuliffe posted. The second variable indicates
a precinct that was adjacent to a precinct in the treatment group (since people in those precincts might be exposed to the signs).  \\

\vspace{.5cm}
\begin{table}[!htbp]
	\centering 
	\textbf{Impact of lawn signs on vote share}\\
	\begin{tabular}{@{\extracolsep{5pt}}lccc} 
		\\[-1.8ex] 
		\hline \\[-1.8ex]
		Precinct assigned lawn signs  (n=30)  & 0.042\\
		& (0.016) \\
		Precinct adjacent to lawn signs (n=76) & 0.042 \\
		&  (0.013) \\
		Constant  & 0.302\\
		& (0.011)
		\\
		\hline \\
	\end{tabular}\\
	\footnotesize{\textit{Notes:} $R^2$=0.094, N=131}
\end{table}

	\newpage
\vspace{.5cm}
\begin{enumerate}
	\item [(a)] Use the results from a linear regression to determine whether having these yard signs in a precinct affects vote share (e.g., conduct a hypothesis test with $\alpha = .05$).
	
	\begin{lstlisting}
	# Option A:
	B1 <- 0.042
	SEb1 <- 0.016
	N <- 131
	
	# Calculate the test statistic
	test_statistic <- B1 / SEb1
	
	# Degrees of freedom
	df <- N - 1
	
	# Two-tailed test, so multiply by 2
	p_value <- 2 * pt(-abs(test_statistic), df)
	
	# Compare p-value to significance level (e.g., 0.05)
	if (p_value < 0.05) {
		print("Reject the null hypothesis: Having yard signs in a precinct affects vote share.")
	} else {
		print("Fail to reject the null hypothesis: No evidence that yard signs affect vote share.")
	}
	\end{lstlisting}
	
	H0: B1 equal to 0
	
	H1: B1 non-zero
	
	B1 = 0.042
	
	SE(B1) = 0.016
	
	t-statistic
	t1= B1/SE(B1) = 0.042/0.016 = 2.625
	
	Significance level  $\alpha = .05$ for a two-tailed test. With N=131 observations
	
	Degrees of freedom DF = N - 3
	
	DF = 131-3 = 128
	
	As t1 (2.625) is greater than +-1.978, we reject the null hypothesis. The results suggest that
	the yard signs appear to influence the vote share in the precincts. 
	
	\newpage		
	\item [(b)]  Use the results to determine whether being
	next to precincts with these yard signs affects vote
	share (e.g., conduct a hypothesis test with $\alpha = .05$).

\begin{lstlisting}
# Option B:
B2 <- 0.042
SEb2 <- 0.013

# Calculate the test statistic
test_statistic <- B2 / SEb2

# Degrees of freedom
df <- N - 1

# Two-tailed test, so multiply by 2
p_value <- 2 * pt(-abs(test_statistic), df)

# Compare p-value to significance level (e.g., 0.05)
if (p_value < 0.05) {
	print("Reject the null hypothesis: Being next to precincts with yard signs affects vote share.")
} else {
	print("Fail to reject the null hypothesis: No evidence that being adjacent to yard signs affects vote share.")
}
\end{lstlisting}
	H0: B2 equal to 0
	
	H1: B2 non-zero
	
	B2 = 0.042
	
	SE(B2) = 0.013
	
	t-statistic
	t2= B2/SE(B2) = 0.042/0.013 = 3.23
	
	Significance level  $\alpha = .05$ for a two-tailed test. With N=131 observations
	
	Degrees of freedom DF = N - 3
	
	DF = 131-3 = 128
	
	As t2 (3.23) is greater than +-1.978, we reject the null hypothesis. 
	
	This suggest that being next to precincts with yard signs has a statistically significance effect on vote share.
	
	\newpage	
	\vspace{1cm}
	\item [(c)] Interpret the coefficient for the constant term substantively.
	
	The constant term or B0 is the estimated value of the dependent variable 
	(proportion of the vote that went to McAuliff's opponent Ken Cuccinelli) when all 
	independent variables are set to zero.
	
	As B0 = 0.302 The starting point for the vote share when there are no yard signs present is 30.2%
	
	\vspace{1cm}
	
	\item [(d)] Evaluate the model fit for this regression.  What does this	tell us about the importance of yard signs versus other factors that are not modeled?
	
	R2 = 0.094 represents the proportion of the variance in the dependent variable that is explained by
	the independent variables in the model. 
	
	In other words 9.4 percent of the variance in the vote share is explained by the variables included
	in the model (B1 Assigned yard signs and B2 Adjacent to yard signs). The other 90.6 percent is not 
	explained by the variables in the model. This could mean that there are other factors (variables) that
	influence the vote share. 
	
\end{enumerate}  


\end{document}
